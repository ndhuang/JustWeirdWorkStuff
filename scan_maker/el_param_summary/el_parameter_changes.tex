\documentclass{article}
\usepackage[margin=1in]{geometry}
\usepackage[dvipdfm]{graphicx}
\usepackage{amsmath}
\usepackage{caption}
\usepackage{subcaption}
\newcommand{\figtitle}[1]{
  {
    \centering
    \large\textbf{#1}
    \par\medskip
  }
}

\begin{document}
\section{Brief Background}
On 2014-05-04, we ran a set of tests to probe changes to control loop 
parameters (telescope\_el\_tests\_05May2014.sch).  The tests were a subset of 
those run on 2014-04-30 (telescope\_el\_tests\_30Apr2014.sch).  
The schedule runs
a series of el steps of various size and duration (the duration is defined as 
the time taken to step up or down).  They are all performed with
constant jerk.  The following table shows the step sizes and durations.  All
possible permutations were run.  For the larger steps and shorter times, a
maximum acceleration was applied that made the actual time larger.  Each scan 
is run 5 times.

\begin{tabular}{|r|c|c|c|c|c|c|}
  \hline 
  Step Size (arcmin) & 3.5 & 6.0 & 12.0 & 20.0 & 40.0 & 80.0 \\
  \hline
  Duration (seconds) & 0.5 & 1.0 & 2.0 & 4.0 & 8.0  &  \\
  \hline
\end{tabular}

We ran this schedule 4 times, with different values for the control loop:

\begin{tabular}{|r|c|c|c|c|}
  \hline
   & Test 1 (normal) & Test 2 (10\% reduction) & 
  Test 3 (20\% reduction) & Test 4 (30\%) \\
  \hline
  El Lead Break & 0.10 & 0.09 & 0.08 & 0.07 \\
  El Crossover Freq & 0.450 & 0.405 & .0360 & 0.315 \\
  El Lag Break & 2.0 & 1.8 & 1.6 & 1.4 \\
  \hline
\end{tabular}

This leaves us with 4 sets of 5 for each scan.  To get some useful information
out of this, I stacked the scans, and then differenced them.

\section{Method}
%% Each set of 5 scans (taken with the same control loop parameters) is stacked,
%% such that the sum of errors squared is minimized.  More concretely the sum

%% is minimized.  $\vec{scan}_3$ is an arbitrary choice, but it works well enough.

Each set of 5 scans (run with the same control loop parameters) is temporally 
aligned.  The method I use to do this leads to a maximum rms error of less than
10 arcmin in most scans.  There are some that fail spectacularly, likely due to
either glitches or other single events in one scan.  I have not included
any data from these scans.

\subsection{Details of the Alignment}
To correctly align the scans, I give each scan a temporal shift with respect to
an arbitrarily chosen scan.  To be specific, I minimize the sum
\begin{equation}
  \sum_{i=1}^{5} \left(\vec{\mathrm{scan}}_{match} - \vec{\mathrm{scan}}_i\right)^2
\end{equation}
where scan$_i$ has been shifted $n$ points relative to scan$_{match}$.
By minimizing this sum, I align 4 of the scans to the arbitrarily chosen 
``match scan''.

\section{Results}
These tests seem to tell us that our current control loop parameters are better
than the any of the changes we tried.  In a few cases, the relaxed parameters
give a lower overshoot, but the total settling time is longer.

We can also see that some steps cause oscillations on top of the overshoot
(see test2\_del-6p0ams-8p0s on page~\pageref{oscillations}).  
This only happens in the smaller steps, but there does not appear to be 
any further trend.
\begin{figure}
    \centering
    \figtitle{Scan test2\_del-3p5ams-4p0s Position}
    \includegraphics[height=.4\textheight]{test2_del-3p5ams-4p0s_pos.eps}
\end{figure}
\begin{figure}
    \centering
    \figtitle{Scan test2\_del-3p5ams-4p0s Error}
    \includegraphics[height=.4\textheight]{test2_del-3p5ams-4p0s_err.eps}
\end{figure}
\begin{figure}
    \centering
    \figtitle{Scan test2\_del-3p5ams-2p0s Position}
    \includegraphics[height=.4\textheight]{test2_del-3p5ams-2p0s_pos.eps}
\end{figure}
\begin{figure}
    \centering
    \figtitle{Scan test2\_del-3p5ams-2p0s Error}
    \includegraphics[height=.4\textheight]{test2_del-3p5ams-2p0s_err.eps}
\end{figure}
\begin{figure}
    \centering
    \figtitle{Scan test2\_del-3p5ams-8p0s Position}
    \includegraphics[height=.4\textheight]{test2_del-3p5ams-8p0s_pos.eps}
\end{figure}
\begin{figure}
    \centering
    \figtitle{Scan test2\_del-3p5ams-8p0s Error}
    \includegraphics[height=.4\textheight]{test2_del-3p5ams-8p0s_err.eps}
\end{figure}
\begin{figure}
    \centering
    \figtitle{Scan test2\_del-6p0ams-1p0s Position}
    \includegraphics[height=.4\textheight]{test2_del-6p0ams-1p0s_pos.eps}
\end{figure}
\begin{figure}
    \centering
    \figtitle{Scan test2\_del-6p0ams-1p0s Error}
    \includegraphics[height=.4\textheight]{test2_del-6p0ams-1p0s_err.eps}
\end{figure}
\begin{figure}
    \centering
    \figtitle{Scan test2\_del-6p0ams-4p0s Position}
    \includegraphics[height=.4\textheight]{test2_del-6p0ams-4p0s_pos.eps}
\end{figure}
\begin{figure}
    \centering
    \figtitle{Scan test2\_del-6p0ams-4p0s Error}
    \includegraphics[height=.4\textheight]{test2_del-6p0ams-4p0s_err.eps}
\end{figure}
\clearpage
\begin{figure}
    \centering
    \figtitle{Scan test2\_del-6p0ams-2p0s Position}
    \includegraphics[height=.4\textheight]{test2_del-6p0ams-2p0s_pos.eps}
\end{figure}
\begin{figure}
    \centering
    \figtitle{Scan test2\_del-6p0ams-2p0s Error}
    \includegraphics[height=.4\textheight]{test2_del-6p0ams-2p0s_err.eps}
\end{figure}
\begin{figure}
    \centering
    \figtitle{Scan test2\_del-6p0ams-8p0s Position}
    \includegraphics[height=.4\textheight]{test2_del-6p0ams-8p0s_pos.eps}
\end{figure}
\begin{figure}
    \centering
    \figtitle{Scan test2\_del-6p0ams-8p0s Error}
    \includegraphics[height=.4\textheight]{test2_del-6p0ams-8p0s_err.eps}
    \label{oscillations}
\end{figure}
\begin{figure}
    \centering
    \figtitle{Scan test2\_del-12p0ams-8p0s Position}
    \includegraphics[height=.4\textheight]{test2_del-12p0ams-8p0s_pos.eps}
\end{figure}
\begin{figure}
    \centering
    \figtitle{Scan test2\_del-12p0ams-8p0s Error}
    \includegraphics[height=.4\textheight]{test2_del-12p0ams-8p0s_err.eps}
\end{figure}
\begin{figure}
    \centering
    \figtitle{Scan test2\_del-12p0ams-4p0s Position}
    \includegraphics[height=.4\textheight]{test2_del-12p0ams-4p0s_pos.eps}
\end{figure}
\begin{figure}
    \centering
    \figtitle{Scan test2\_del-12p0ams-4p0s Error}
    \includegraphics[height=.4\textheight]{test2_del-12p0ams-4p0s_err.eps}
\end{figure}
\clearpage
\begin{figure}
    \centering
    \figtitle{Scan test2\_del-20p0ams-4p0s Position}
    \includegraphics[height=.4\textheight]{test2_del-20p0ams-4p0s_pos.eps}
\end{figure}
\begin{figure}
    \centering
    \figtitle{Scan test2\_del-20p0ams-4p0s Error}
    \includegraphics[height=.4\textheight]{test2_del-20p0ams-4p0s_err.eps}
\end{figure}
\begin{figure}
    \centering
    \figtitle{Scan test2\_del-20p0ams-8p0s Position}
    \includegraphics[height=.4\textheight]{test2_del-20p0ams-8p0s_pos.eps}
\end{figure}
\begin{figure}
    \centering
    \figtitle{Scan test2\_del-20p0ams-8p0s Error}
    \includegraphics[height=.4\textheight]{test2_del-20p0ams-8p0s_err.eps}
\end{figure}
\begin{figure}
    \centering
    \figtitle{Scan test2\_del-20p0ams-2p0s Position}
    \includegraphics[height=.4\textheight]{test2_del-20p0ams-2p0s_pos.eps}
\end{figure}
\begin{figure}
    \centering
    \figtitle{Scan test2\_del-20p0ams-2p0s Error}
    \includegraphics[height=.4\textheight]{test2_del-20p0ams-2p0s_err.eps}
\end{figure}
\begin{figure}
    \centering
    \figtitle{Scan test2\_del-40p0ams-8p0s Position}
    \includegraphics[height=.4\textheight]{test2_del-40p0ams-8p0s_pos.eps}
\end{figure}
\begin{figure}
    \centering
    \figtitle{Scan test2\_del-40p0ams-8p0s Error}
    \includegraphics[height=.4\textheight]{test2_del-40p0ams-8p0s_err.eps}
\end{figure}
\clearpage
\begin{figure}
    \centering
    \figtitle{Scan test2\_del-80p0ams-4p0s Position}
    \includegraphics[height=.4\textheight]{test2_del-80p0ams-4p0s_pos.eps}
\end{figure}
\begin{figure}
    \centering
    \figtitle{Scan test2\_del-80p0ams-4p0s Error}
    \includegraphics[height=.4\textheight]{test2_del-80p0ams-4p0s_err.eps}
\end{figure}
\begin{figure}
    \centering
    \figtitle{Scan test2\_del-80p0ams-8p0s Position}
    \includegraphics[height=.4\textheight]{test2_del-80p0ams-8p0s_pos.eps}
\end{figure}
\begin{figure}
    \centering
    \figtitle{Scan test2\_del-80p0ams-8p0s Error}
    \includegraphics[height=.4\textheight]{test2_del-80p0ams-8p0s_err.eps}
\end{figure}
\clearpage

\end{document}
